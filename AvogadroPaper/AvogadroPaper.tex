\documentclass[10pt]{bmc_article}

% Load packages
\usepackage{cite} % Make references as [1-4], not [1,2,3,4]
\usepackage{url}  % Formatting web addresses
\usepackage{ifthen}  % Conditional
\usepackage{multicol}   %Columns
\usepackage[utf8]{inputenc} %unicode support
\urlstyle{rm}

%%%%%%%%%%%%%%%%%%%%%%%%%%%%%%%%%%%%%%%%%%%%%%%%%
%%                                             %%
%%  If you wish to display your graphics for   %%
%%  your own use using includegraphic or       %%
%%  includegraphics, then comment out the      %%
%%  following two lines of code.               %%
%%  NB: These line *must* be included when     %%
%%  submitting to BMC.                         %%
%%  All figure files must be submitted as      %%
%%  separate graphics through the BMC          %%
%%  submission process, not included in the    %%
%%  submitted article.                         %%
%%                                             %%
%%%%%%%%%%%%%%%%%%%%%%%%%%%%%%%%%%%%%%%%%%%%%%%%%
\usepackage{graphicx}
%\def\includegraphic{}
%\def\includegraphics{}

\setlength{\topmargin}{0.0cm}
\setlength{\textheight}{21.5cm}
\setlength{\oddsidemargin}{0cm}
\setlength{\textwidth}{16.5cm}
\setlength{\columnsep}{0.6cm}

\newboolean{publ}

%%%%%%%%%%%%%%%%%%%%%%%%%%%%%%%%%%%%%%%%%%%%%%%%%%
%%                                              %%
%% You may change the following style settings  %%
%% Should you wish to format your article       %%
%% in a publication style for printing out and  %%
%% sharing with colleagues, but ensure that     %%
%% before submitting to BMC that the style is   %%
%% returned to the Review style setting.        %%
%%                                              %%
%%%%%%%%%%%%%%%%%%%%%%%%%%%%%%%%%%%%%%%%%%%%%%%%%%

%Review style settings
\newenvironment{bmcformat}{\begin{raggedright}
\baselineskip20pt\sloppy\setboolean{publ}{false}}{\end{raggedright}
\baselineskip20pt\sloppy}

%Publication style settings
%\newenvironment{bmcformat}{\fussy\setboolean{publ}{true}}{\fussy}

%New style setting
%\newenvironment{bmcformat}{\baselineskip20pt\sloppy\setboolean{publ}{false}}{
%\baselineskip20pt\sloppy}

% Begin ...
\begin{document}
\begin{bmcformat}


%%%%%%%%%%%%%%%%%%%%%%%%%%%%%%%%%%%%%%%%%%%%%%
%%                                          %%
%% Enter the title of your article here     %%
%%                                          %%
%%%%%%%%%%%%%%%%%%%%%%%%%%%%%%%%%%%%%%%%%%%%%%

\title{Avogadro: A Framework and Cross Platform GUI for Building Molecular
  Structures and the Analysis of Output}

%%%%%%%%%%%%%%%%%%%%%%%%%%%%%%%%%%%%%%%%%%%%%%
%%                                          %%
%% Enter the authors here                   %%
%%                                          %%
%% Ensure \and is entered between all but   %%
%% the last two authors. This will be       %%
%% replaced by a comma in the final article %%
%%                                          %%
%% Ensure there are no trailing spaces at   %%
%% the ends of the lines                    %%
%%                                          %%
%%%%%%%%%%%%%%%%%%%%%%%%%%%%%%%%%%%%%%%%%%%%%%
\author{Marcus D Hanwell\correspondingauthor$^{1, 2}$%
  \email{Marcus D Hanwell\correspondingauthor - marcus.hanwell@kitware.com}
  \and
  Donald E Curtis$^3$%
  \and
  David Lonie$^4$%
  \and
  Tim Vandermeersch$^5$%
  \and
  Eva Zurek$^4$%
  \and
  Geoffrey R Hutchison$^1$%
  \email{Geoffrey R Hutchison - geoffh@pitt.edu}
  }

%%%%%%%%%%%%%%%%%%%%%%%%%%%%%%%%%%%%%%%%%%%%%%
%%                                          %%
%% Enter the authors' addresses here        %%
%%                                          %%
%%%%%%%%%%%%%%%%%%%%%%%%%%%%%%%%%%%%%%%%%%%%%%

\address{%
  \iid(1)Department of Chemistry, University of Pittsburgh, 219 Parkman Avenue,
Pittsburgh, PA, 15260, USA\\
  \iid(2)Kitware, Inc., 28 Corporate Drive, Clifton Park, NY, 12065, USA\\
  \iid(3)Department of Computer Science, University of Iowa\\
  \iid(4)Department of Chemistry, State University of New York at Buffalo\\
  \iid(5)Tim's Affiliation, Europe
}%

\maketitle

%%%%%%%%%%%%%%%%%%%%%%%%%%%%%%%%%%%%%%%%%%%%%%
%%                                          %%
%% The Abstract begins here                 %%
%%                                          %%
%% Please refer to the Instructions for     %%
%% authors on http://www.biomedcentral.com  %%
%% and include the section headings         %%
%% accordingly for your article type.       %%
%%                                          %%
%%%%%%%%%%%%%%%%%%%%%%%%%%%%%%%%%%%%%%%%%%%%%%

\begin{abstract}

Avogadro is a free, cross-platform, open source, OpenGL based graphical user
interface and library for building molecular structures, formatting input files
and analyzing output files from computational chemistry codes. The work
presented here details the Avogadro library, which provides a framework,
application programming interface, and three-dimensional visualization
capabilities that have direct applications in research and education in the
fields of chemistry, physics, materials science and biology. The Avogadro
application provides a rich graphical interface using dynamically loaded
plugins. The application can be extended by implementing a plug-in module in C++
or Python in order to explore different visualization techniques,
build/manipulate molecular structures, and interact with other programs.
We describe some example extensions, one which uses a genetic algorithm to find
stable crystal structures, and one which interfaces with the PackMol program to
create packed, solvated structures for molecular dynamics simulations.

\end{abstract}

\ifthenelse{\boolean{publ}}{\begin{multicols}{2}}{}

\section{Introduction}

Many areas such as chemistry, materials science, physics and biology need
efficient computer programs to both build and visualize molecular structures.
The field of molecular graphics is dominated by viewers with little or no
editing capabilities, such as RasMol~\cite{RasMol}, JMol~\cite{JMol},
PyMOL~\cite{PyMOL}, VMD~\cite{VMD}, QuteMol~\cite{QuteMol}, and
BALLView~\cite{BALLView} among many others. The aforementioned viewers are all
freely available, and most of them are available under open source licenses and
work on the most common operating systems (GNU/Linux, Apple Mac OS X, BSD and
Microsoft Windows).

The choice of software capable of building chemical structures is far smaller.
There are existing commercial packages, such as Spartan, CAChe, GaussView,
Materials Studio~\cite{Accelrys} and CrystalMaker~\cite{CrystalMaker}, that are
well polished and capable of constructing many different types of molecular
structures. They are however not available for all operating systems (most of
them only have programs for Microsoft Windows), are not easily extensible or
released under an open source license. Licensing costs can be prohibitive, as
can the changing direction of companies which can lead to a loss of significant
research investment in a single commercial product.

The selection of free, open source, cross platform molecular builders was quite
limited when the Avogadro project was founded in late 2006.
Ghemical~\cite{Ghemical} was one of the only projects satisfying these needs at
the time. Two of the authors (Hutchison and Curtis) contributed to Ghemical
previously, but had found that it was not easily extensible. This led them to
found a new project in order to address the issues they had observed in Ghemical
and other packages. The Molden~\cite{Molden} application was also available,
able to build up small molecules and analyze output from several quantum codes.
It suffers from a restrictive license and also uses an antiquated graphical
toolkit.

Following a case study of the available applications capable of building
molecular structures broad design goals were outlined. One of the main issues
with both commercial and open source applications is a lack of extensibility,
many of the applications also only work on one or two operating systems. An open
and extensible framework is needed in order to be able to perform innovative
research. The creation of a framework that implements many of the necessary
foundations for a molecular builder and visualizer would facilitate more
effective research in this area in the future.

At the time of writing it is apparent that other researchers have perceived
similar needs. Several new applications are available today that focus on both
building and visualizing molecular structure. These include
CCP1GUI~\cite{CCP1GUI}, Gabedit~\cite{Gabedit} along with some highly specific
editors such as MacMolPlt~\cite{MacMolPlt}. Whilst offering many interesting and
useful features, these projects suffer from the same kind of issues centering
around effective reuse of existing code, well commented and documented code and
easy extension to specialized areas.

The Avogadro project has endeavored to make a free, open source framework for
both building and visualizing molecular structures. Much of the initial focus
has been placed on preparing input and analyzing output from quantum
calculations. Other applications such as preparing input for MD simulations and
visualizing periodic structures will also be presented, demonstrating the
flexibility of the Avogadro framework.

Avogadro has close ties to several other free, cross platform, open source
projects in order to reuse as much code as is practical. These projects include
Nokia Qt to provide a free, cross platform graphical toolkit, Open
Babel~\cite{OpenBabel} for chemical file input/output, geometry optimization and
other chemical perception, Eigen~\cite{Eigen} for matrix and vector mathematics,
OpenGL/GLSL for real time three dimensional rendering and POV-Ray for ray-traced
rendering.

Avogadro also uses quantum codes such as GAMESS-US, Q-Chem, Gaussian and MOPAC
to perform ab-initio and semi-empirical calculations. Support for more
calculation backends is planned and we are working with Q-Chem on an advanced
input file generator. As with most open source projects, the directions the
project has taken were influenced largely by the research areas and interests of
the principal contributors.

\section{The Graphical User Interface}

%The first thing most people will see is the main Avogadro application window.
Binary installers are provided under the GPL license for Apple Mac OS X and
Microsoft Windows, along with packages for all of the major Linux distributions.
This means that Avogadro can be installed quite easily on most operating
systems. Easy to follow instructions on how to compile the latest source code
are also provided on the main Avogadro web site for the more adventurous, or
those using an operating system that is not yet supported.

The Qt toolkit from Nokia gives Avogadro a native look and feel on the three
supported operating systems---Linux, Mac OS X and Windows. The basic
functionality expected in a molecular builder and viewer has been implemented,
along with several less common features. The vast majority of this functionality
has been written using the API made available to plug-in writers, and is loaded
at runtime. It is very easy for new users to install Avogadro and build their
first molecules within minutes. Thanks to the Open Babel library Avogadro
supports a large portion of the chemical file formats that are in common use.

The first thing most people will see is the main Avogadro application window.
The basic functionality expected in a molecular builder and viewer has been
implemented, along with several less common features. The vast majority of this
functionality has been written using the API made available to plugin writers,
and is loaded at runtime. It is very easy for new users to install Avogadro and
build their first molecules within minutes. Thanks to the Open Babel library
Avogadro supports a large portion of the chemical file formats that are in
common use.

\begin{figure}
  \begin{center}
    \includegraphics[width=0.8\textwidth]{images/avogadro-drawing}
  \end{center}
  \caption{The Avogadro graphical user interface, showing the editing interface
for a molecule.}
  \label{f:avogadrogui}
\end{figure}


\subsection{Building a Molecule: Atom by Atom}

After opening Avogadro a window such as that shown in figure~\ref{f:avogadrogui}
is presented. From there simply left clicking on the black part of the display
allows the user to draw a carbon atom. If the user pushes the left mouse button
down and drags a bonded carbon atom would be drawn between the start point and
the final position where the mouse was released.

A large amount of effort has been expended to create an intuitive tool for
drawing small molecules. Common elements can be selected from a drop down list,
or a periodic table can be displayed to select less common elements. Clicking on
an existing atom changes its type, dragging reverts the original atom and draws
the new atom bonded to the original. If the bonds are left-clicked then the bond
order cycles between single, double and triple.

Right clicking on atoms or bonds deletes them. If the ``Adjust Hydrogens'' box
is checked the number of hydrogens bonded to each atom is automatically adjusted
to satisfy valency. This can also be done at the end of an editing session by
using the add hydrogens extension in the build menu. Keyboard shortcuts are also
available to change the active element (by typing the one or two letter symbol),
or bond order (by entering the numeric bond order).

In addition to the draw tool there are two tools for adjusting the position of
atoms in existing molecules. The atom centric manipulate tool can be used to
move an atom or a group of selected atoms. The bond centric manipulate tool can
be used to select a bond and then adjust all atoms positions relative to the
selected bond in various ways. These three tools allow for a great deal of
flexibility in building small molecules interactively on screen.

Once the molecular structure is complete the forcefield extension can be used to
perform a geometry optimization. By clicking on ``Extensions'' and ``Optimize
Geometry'' a fast geometry optimization is performed on the molecule. The
forcefield and calculation parameters can be adjusted, but the defaults are
adequate for most molecules. This workflow is typical when building up a small
molecular structures for use as input to quantum calculations, or publication
quality figures.

An alternative is to combine the ``Auto Optimization'' tool with the drawing
tool. This presents a unique way of sculpting the molecule while the geometry is
constantly minimized in the background. The geometry optimization is animated,
and the effect of changing bond orders, adding new groups or removing groups can
be observed interactively.

Several dialogs are provided to provide information on molecule properties, and
to precisely change parameters, such as the cartesian coordinates of the atoms
in the molecule.

\subsection{Building a Molecule: From Fragments} %Geoff?

\begin{figure}
  \includegraphics[width=0.44\textwidth]{images/avogadro-q-chem}
  \hspace{0.1cm}
  \includegraphics[width=0.44\textwidth]{images/avogadro-gaussian}
  \caption{Dialog for generating input for Q-Chem (left) and Gaussian (right).
    Note that both dialogs are similar in interface, allowing users to use
    multiple computational chemistry packages.}
  \label{f:quantumdialogs}
\end{figure}

\subsection{Preparing Input for Quantum Codes}

Several extensions were developed for Avogadro that assist the user in preparing
input files for popular quantum codes such as GAMESS-US, Gaussian, Q-Chem and
MOPAC. The graphical dialogs present the features required to run basic quantum
calculations, some examples are shown in figure~\ref{f:quantumdialogs}.

The preview of the input file at the bottom of each dialog is updated as options
are changed. This approach helps new users of quantum codes to learn the syntax
of input files for different codes, and to quickly generate useful input files
as they learn. The input can be edited in the dialog, before the file is saved
and submitted to the quantum code. The MOPAC extension can also run the  MOPAC
program directly if it is available on the user's computer, Avogadro can also
load the output file once the calculation is complete.

\subsection{Alignment and Measurements}

One of the specialized tools included in the standard Avogadro distribution is
the alignment tool. This mouse tool facilitates the alignment of a molecular
structure with the coordinate origin if one atom is selected, and along the
specified axis if two atoms are selected. The alignment tool can be combined
with the measure, select and manipulate tools in order to create inputs for
quantum codes where the position and orientation of the molecule is important.
One example of this is calculations where an external electric field is applied
to the molecule. In these types of calculations the alignment of the molecule
can have a large effect.

More complex alignment tools for specific tasks couuld be created. The alignment
tool was created in just a few hours for a specific research project performing
calculation on the piezoelectric effect in single molecules. This is a prime
example where extensibility was very important in order to be able to perform
research using a graphical computational chemistry tool. It would not have been
worth the investment to create a new application just to align molecular
structures to an axis, but creating a plugin for an extensible project was not
unreasonable.


\section{Visualization}

The Avogadro application uses OpenGL to render molecular representations to the
screen interactively. This has the advantage of being a good cross platform API
to render three dimensional images using hardware accelerated graphics. OpenGL
1.1 and below is used in most of the rendering code, and so Avogadro can be used
even on very modest/old computer systems. It is capable of taking advantage of
some of the newer features available in OpenGL 2.0, but this has been kept as an
optional extra feature when working on novel visualizations of molecular
structure.

\subsection{Standard Representations}

\begin{figure}
  \includegraphics[width=0.95\textwidth]{images/standardRepsLabel}
  \caption{Several molecular representations of thiophene, (a) wireframe,
    (b) stick/liquorice, (c) ball and stick, (d) ball and stick with ring,
    (e) Van der Waal's/CPK and (f) transparent Van der Waal's with stick.}
  \label{f:standardReps}
\end{figure}

In chemistry there are several standard representations of molecular structre,
originally based upon those possible with physical models. The Avogadro
application implements each of these representations shown in
figure~\ref{f:standardReps} as a plugin. These range from the simple wireframe
representation, stick/liquorice, ball and stick and Van der Waal's.

It is also possible to combine several representations, such as ball and stick
with ring rendering, figure~\ref{f:standardReps} (d), and a semi-transparent Van
der Waal's space-filling representation with a stick representation to eludicate
molecular backbone, figure~\ref{f:standardReps} (f).

\subsection{Electronic Structure}

Avogadro can read Gaussian cube files, in addition it can also read the basis
set and relevant information output by Gaussian, Q-Chem, Molpro and MOPAC. The
volumetric data can be calculated and isosurfaces computed from that volumetric
data. There are display types that can display these isosurfaces in Avogadro
interactively.

\subsection{Secondary Biological Structure} % This would likely be a great one
for Tim to add some detail to.

\subsection{GLSL, Novel Visualization}



\subsection{Ray Tracing}

Description and examples of ray traced images.

\section{Software Architecture}

One area that seems to suffer in many code bases in chemistry is software
architecture. This can lead to less maintainable code, poor code reuse and a
much higher barrier to entry. Problems were identified in other projects with a
view to minimizing their impact when developing Avogadro. Modern software design
processes were used in the initial planning stages of Avogadro, along with the
choice of modern programming languages and libraries.

Based on the previous experience of the authors, and a review of available
programs at the time, several fundamental choices were made. The C++ programming
language was chosen, with Qt as the cross platform graphical toolkit, OpenGL for
3D rendering, CMake as the build system and Open Babel as the chemical library.
Using this combination of languages and libraries allowed for the project to be
licensed under the GNU GPLv2 license and made (and kept) openly available to
all.

The choice of license is an interesting one, and is hotly debated in many
industries. The choice of the GPLv2 was necessitated because of the use of the
Qt and Open Babel libraries. Qt has since been released under the much more
permissive LGPL license. Packages such as PyMol use the more permissive BSD
license, selling commercial versions in addition to the open source version
(that has some capabilities missing that are present in the commercial version).
The GPLv2 license is a good choice in order to ensure that the code base remains
free and open, philosophically blocking the commercialization of the code at no
cost to the company. This allows for open collaboration but precludes some
avenues for funding further development.

The core of Avogadro is written in portable C++ code with platform specific
differences abstracted away by Qt, OpenGL and Open Babel. The CMake build system
makes the build process relatively simple on all supported platforms. Avogadro
has been successfully built and tested on x86 and x86\_64 versions of Linux, PPC
and x86 version of Apple Mac OS X and x86 Microsoft Windows. By leveraging the
power of existing libraries the time required to develop new features has been
significantly reduced.

Avogadro has a well defined set of interfaces and programming interfaces. Almost
all functionality is implemented in self contained plugins that are loaded at
runtime. The majority of these plugins are written in C++ but the Avogadro API
has also been exposed to the Python scripting language. This allows for a great
deal of choice in how plugins are implemented. In both cases each plugin is a
self contained class that implements a set of functions that are part of the
Avogadro API, allowing for a wide variety of features to be implemented in a
very modular way.

The Avogadro framework uses the model, view, controller paradigm. The model
being the core data classes such as Molecule, Atom and Bond, the view being the
engine plugins and the controllers being the tools (interactive mouse) and
extensions (non-interactive, form based/menu based). Each plugin has full access
to the core data model.

\subsection{Plug-in Interface}

The Avogadro library was developed as an extensible library using C++ plugins
that are loaded at runtime for most functionality. The Avogadro plugins are
divided into four separate types, all having a common base class. The Plugin
class is the common base, defining a minimal set of interfaces for an Avogadro
plugin inside the Avogadro C++ namespace.

There are four classes that derive from this common base class specializing
their interface for specific activities. The Avogadro::Color base class defines
the virtual interface for applying colors to atoms, bonds and other properties.
Avogadro::Engine defines the common interface for all display types in Avogadro,
from the simple ball and stick or Van der Waals visualizations through to
surfaces and force visualizations. The Avogadro::Tool base class provides
the interface for all interactive tools, focusing principally on mouse and
keyboard interaction with Avogadro. Examples of tool plugins include the draw
tool used to draw molecules atom by atom, and the navigation tool used to pan,
rotate and scale the view of the molecule. There are also several specialized
tools such as the alignment tool.

Finally there is the Avogadro::Extension class, which defines the interface for
dialog based plugins. These extensions can interact with the molecule, and are
used for a variety of purposes from molecule properties dialogs to input file
generation dialogs for many quantum codes including NWChem, Gaussian, GAMESS and
others. This class of plugin is also applied to file import, and network aware
extensions querying web databases for structures given their common name for
example.

When the application starts up it searches several directories for plugins,
which it then attempts to load. The Qt plugin framework is used to check that
the plugins have a recent enough version to be loaded, and the plugin type can
be deduced once it has been loaded. The user interface is then populated with
appropriate entries. The tools being added to the main toolbar using their
embedded icons, display types are added to the display type list, and menu
entries are added for all loaded extensions.

The tool and display type plugins can both (optionally) return a dialog that is
used to configure the plugin. These are specific to each plugin, and provided in
the user interface.

\subsection{Display Types}

Display types are one type of plugin, referred to as engines internally. Their
primary focus is to render graphics to the screen. As is the case with most
molecular graphics a large portion of the geometric primitives are spheres and
cylinders. These are typically used to represent atoms and bonds, but there are
many other properties that can be rendered using the display type plugins.

Some of the engines also convey some information about the underlying data the
geometric primitives represent, in order to allow for the molecule to be edited.
Engines are performance critical as the render functions are called each time a
frame is requested for display. It is very important for the engines to
efficiently render all requested data, and multiple display types can be
combined to form a composite display, for example ball and stick display
overlaid with a transparent Van der Waals space filling diplay and ring
rendering to highlight all rings in the structure. Figure~\ref{f:standardReps}
(d) and (f) show two such combinations of multiple display types.

\subsection{Tools}

The tools are responsible for virtually all mouse and keyboard interaction with
the molecule. The navigation tool provides basic scene navigation, implementing
rotation, panning, tilting and zooming support. This tool provides optional
visual cues to show what type of navigation is taking place, and all scene
navigation will take place about the center of the molecule if an empty space is
clicked, or about the center of the clicked atom. The navigation tool is also
used as the default tool if the currently active tool does not handle the mouse
event passed to it.

One of the other central tools is the draw tool, this implements a free hand
molecule drawing input method, supporting keyboard shortcuts, combo boxes or a
periodic table view to select elements. The user can use the left mouse button
to add new atoms or bonds, or click on the bonds to change their order. The
right mouse button can be used to delete atoms or bonds and the directional keys
can be used in combination with the mouse to quickly rotate/pan the molecule.

There are also two tools for adjustment of structures (atom or bond centric), a
selection tool supporting standard selection interactions and an auto-rotate
tool that allows users to set the speed and angles about which to rotate the
molecule. The interactive auto-optimization tool provides a sculpting
interaction, where the user can begin a continuous geometry optimization, and
switch back to the draw or adjustment tools and change the shape and structure
of the molecule while observing the new structure being optimized.

This can also be combined with the measurement tool to interactively observe
bond lengths and angles evolve as the structure is updated and the geometry
minimized. If the optimization tool is turned off the measurement tool also
allows the user to precisely adjust bond lengths and/or angles using the
adjustment tools.

\subsection{Extensions}

The extensions represent quite a diverse range of plugins. These range from the
input generation dialogs for various quantum chemistry codes such as GAMESS,
MOLPRO, NWChem, etc, through to animation of the molecule and visualization of
molecular orbitals and electron density. Network aware extensions allow the user
to click on File->Import->Fetch by chemical name and search for ``tnt'' or
``propanol'' and have structures returned by the NIH CACTUS Chemical Structure
Resolver service.

Other extensions translate the entire scene to POV-Ray input, and call POV-Ray
to render the molecule using ray tracing techniques to provide higher quality
renderings for publication. Various molecular property dialogs are also
implemented as plugins, drawing largely on Open Babel functionality to provide
an overview of the molecule. Cartesian editors, addition and removal of
hydrogens, fragment, SMILES and peptide insertion are all implemented as
extensions showing up in the Avogadro menus. More recently a crystallography
extension was added giving access to a much wider range to functionality
useful to practitioners in that area.

\subsection{Colors} % Geoff

\section{Python Interface} % Tim

Description of the Python bindings, PyQt, plugins as Python classes.

\section{Quantum Calculations}

Description of the quantum calculation code already implemented. Slater and
Gaussian types, output file parsing, parallel calculation code and details of
data layout.

\section{Avogadro Library in Use}

The Avogadro library used in Kalzium, may be Kompound if there is more to show.
QUI as an external user, not yet well integrated.

\subsection{Packmol}

\begin{figure}
  \includegraphics[width=0.45\textwidth]{images/packmol-extension}
  \includegraphics[width=0.45\textwidth]{images/packmol-lipid}
  \caption{The PackMol extension for Avogadro (left) and a lipid layer (right)
    as produced by the PackMol program.}
  \label{f:packmol}
\end{figure}

\subsection{XtalOpt}

\begin{figure}
  \includegraphics[width=0.95\textwidth]{images/xtalopt}
  \caption{The XtalOpt package showing a plot of stability vs. search progress
    for a $\mathrm{TiO_2}$ supercell.}
  \label{f:xtalopt}
\end{figure}

The XtalOpt~\cite{xo1, xo2} software package is implemented as a third-party C++
extension to Avogadro and makes heavy use of the libavogadro API. The extension
implements an evolutionary algorithm tailored for crystal structure prediction.
The XtalOpt development team chose Avogadro as a development platform for the
open source license, well-designed API, powerful visualization tools, and
intuitive interface of the Avogadro package. XtalOpt exists as a dialog window
(Figure XO) and uses the main Avogadro window for visualizing candidate
structures as they evolve. The API is well suited for XtalOpt’s needs,
providing a simple mechanism to allow the user to view, edit, and export the
structures generated during the search. Taking advantage of the cross-platform
capabilities of Avogadro and its dependencies, XtalOpt is available for Linux,
Windows, (Mac?).

\section{Conclusions}

\section{Acknowledgements}

We wish to thank the many contributors to the Avogadro project, including
developers, testers, translators, and users. MDH and GRH thank the University
of Pittsburgh for support.

%%%%%%%%%%%%%%%%%%%%%%%%%%%%%%%%%%%%%%%%%%%%%%%%%%%%%%%%%%%%%
%%                  The Bibliography                       %%
%%                                                         %%
%%  Bmc_article.bst  will be used to                       %%
%%  create a .BBL file for submission, which includes      %%
%%  XML structured for BMC.                                %%
%%  After submission of the .TEX file,                     %%
%%  you will be prompted to submit your .BBL file.         %%
%%                                                         %%
%%                                                         %%
%%  Note that the displayed Bibliography will not          %%
%%  necessarily be rendered by Latex exactly as specified  %%
%%  in the online Instructions for Authors.                %%
%%                                                         %%
%%%%%%%%%%%%%%%%%%%%%%%%%%%%%%%%%%%%%%%%%%%%%%%%%%%%%%%%%%%%%

\newpage
{\ifthenelse{\boolean{publ}}{\footnotesize}{\small}
 \bibliographystyle{bmc_article}  % Style BST file
  \bibliography{AvogadroPaper} }     % Bibliography file (usually '*.bib' )

%%%%%%%%%%%

\ifthenelse{\boolean{publ}}{\end{multicols}}{}

\end{bmcformat}
\end{document}
