%% BioMed_Central_Tex_Template_v1.06
%%                                      %
%  bmc_article.tex            ver: 1.06 %
%                                       %

%%IMPORTANT: do not delete the first line of this template
%%It must be present to enable the BMC Submission system to
%%recognise this template!!

%%%%%%%%%%%%%%%%%%%%%%%%%%%%%%%%%%%%%%%%%
%%                                     %%
%%  LaTeX template for BioMed Central  %%
%%     journal article submissions     %%
%%                                     %%
%%         <14 August 2007>            %%
%%                                     %%
%%                                     %%
%% Uses:                               %%
%% cite.sty, url.sty, bmc_article.cls  %%
%% ifthen.sty. multicol.sty		   %%
%%				      	   %%
%%                                     %%
%%%%%%%%%%%%%%%%%%%%%%%%%%%%%%%%%%%%%%%%%


%%%%%%%%%%%%%%%%%%%%%%%%%%%%%%%%%%%%%%%%%%%%%%%%%%%%%%%%%%%%%%%%%%%%%
%%                                                                 %%
%% For instructions on how to fill out this Tex template           %%
%% document please refer to Readme.pdf and the instructions for    %%
%% authors page on the biomed central website                      %%
%% http://www.biomedcentral.com/info/authors/                      %%
%%                                                                 %%
%% Please do not use \input{...} to include other tex files.       %%
%% Submit your LaTeX manuscript as one .tex document.              %%
%%                                                                 %%
%% All additional figures and files should be attached             %%
%% separately and not embedded in the \TeX\ document itself.       %%
%%                                                                 %%
%% BioMed Central currently use the MikTex distribution of         %%
%% TeX for Windows) of TeX and LaTeX.  This is available from      %%
%% http://www.miktex.org                                           %%
%%                                                                 %%
%%%%%%%%%%%%%%%%%%%%%%%%%%%%%%%%%%%%%%%%%%%%%%%%%%%%%%%%%%%%%%%%%%%%%

\NeedsTeXFormat{LaTeX2e}[1995/12/01]
\documentclass[10pt]{bmc_article}

% Load packages
\usepackage{cite} % Make references as [1-4], not [1,2,3,4]
\usepackage{url}  % Formatting web addresses
\usepackage{ifthen}  % Conditional
\usepackage{multicol}   %Columns
\usepackage[utf8]{inputenc} %unicode support
\urlstyle{rm}

%%%%%%%%%%%%%%%%%%%%%%%%%%%%%%%%%%%%%%%%%%%%%%%%%
%%                                             %%
%%  If you wish to display your graphics for   %%
%%  your own use using includegraphic or       %%
%%  includegraphics, then comment out the      %%
%%  following two lines of code.               %%
%%  NB: These line *must* be included when     %%
%%  submitting to BMC.                         %%
%%  All figure files must be submitted as      %%
%%  separate graphics through the BMC          %%
%%  submission process, not included in the    %%
%%  submitted article.                         %%
%%                                             %%
%%%%%%%%%%%%%%%%%%%%%%%%%%%%%%%%%%%%%%%%%%%%%%%%%
%\usepackage{graphicx}
\def\includegraphic{}
\def\includegraphics{}

\setlength{\topmargin}{0.0cm}
\setlength{\textheight}{21.5cm}
\setlength{\oddsidemargin}{0cm}
\setlength{\textwidth}{16.5cm}
\setlength{\columnsep}{0.6cm}

\newboolean{publ}

%%%%%%%%%%%%%%%%%%%%%%%%%%%%%%%%%%%%%%%%%%%%%%%%%%
%%                                              %%
%% You may change the following style settings  %%
%% Should you wish to format your article       %%
%% in a publication style for printing out and  %%
%% sharing with colleagues, but ensure that     %%
%% before submitting to BMC that the style is   %%
%% returned to the Review style setting.        %%
%%                                              %%
%%%%%%%%%%%%%%%%%%%%%%%%%%%%%%%%%%%%%%%%%%%%%%%%%%

%Review style settings
\newenvironment{bmcformat}{\begin{raggedright}
  \baselineskip20pt\sloppy\setboolean{publ}{false}}{\end{raggedright}
  \baselineskip20pt\sloppy}

%Publication style settings
%\newenvironment{bmcformat}{\fussy\setboolean{publ}{true}}{\fussy}

%New style setting
%\newenvironment{bmcformat}{\baselineskip20pt\sloppy\setboolean{publ}{false}}{
%\baselineskip20pt\sloppy}

% Begin ...
\begin{document}
\begin{bmcformat}


%%%%%%%%%%%%%%%%%%%%%%%%%%%%%%%%%%%%%%%%%%%%%%
%%                                          %%
%% Enter the title of your article here     %%
%%                                          %%
%%%%%%%%%%%%%%%%%%%%%%%%%%%%%%%%%%%%%%%%%%%%%%

\title{Avogadro: An Advanced Semantic Chemical Editor, Visualization, and
  Analysis Platform}

%%%%%%%%%%%%%%%%%%%%%%%%%%%%%%%%%%%%%%%%%%%%%%
%%                                          %%
%% Enter the authors here                   %%
%%                                          %%
%% Ensure \and is entered between all but   %%
%% the last two authors. This will be       %%
%% replaced by a comma in the final article %%
%%                                          %%
%% Ensure there are no trailing spaces at   %%
%% the ends of the lines                    %%
%%                                          %%
%%%%%%%%%%%%%%%%%%%%%%%%%%%%%%%%%%%%%%%%%%%%%%
\author{Marcus D Hanwell\correspondingauthor$^{1, 2}$%
  \email{Marcus D Hanwell\correspondingauthor - marcus.hanwell@kitware.com}
  \and
  Donald E Curtis$^3$%
  \and
  David C Lonie$^4$%
  \and
  Tim Vandermeersch$^5$%
  \and
  Eva Zurek$^4$%
  \and
  Geoffrey R Hutchison$^1$%
  \email{Geoffrey R Hutchison - geoffh@pitt.edu}
  }

%%%%%%%%%%%%%%%%%%%%%%%%%%%%%%%%%%%%%%%%%%%%%%
%%                                          %%
%% Enter the authors' addresses here        %%
%%                                          %%
%%%%%%%%%%%%%%%%%%%%%%%%%%%%%%%%%%%%%%%%%%%%%%

\address{%
  \iid(1)Department of Chemistry, University of Pittsburgh, 219 Parkman Avenue,
Pittsburgh, PA, 15260, USA\\
  \iid(2)Department of Scientific Computing, Kitware, Inc., 28 Corporate Drive,
Clifton Park, NY, 12065, USA\\
  \iid(3)Department of Computer Science, Coe College, 1220 First Avenue NE,
Cedar Rapids, Iowa 52402\\
  \iid(4)Department of Chemistry, State University of New York at Buffalo,
Buffalo, New York 14260-3000\\
  \iid(5)Avogadro development team
}%

\maketitle

%%%%%%%%%%%%%%%%%%%%%%%%%%%%%%%%%%%%%%%%%%%%%%
%%                                          %%
%% The Abstract begins here                 %%
%%                                          %%
%% Please refer to the Instructions for     %%
%% authors on http://www.biomedcentral.com  %%
%% and include the section headings         %%
%% accordingly for your article type.       %%
%%                                          %%
%%%%%%%%%%%%%%%%%%%%%%%%%%%%%%%%%%%%%%%%%%%%%%

\begin{abstract}

\textbf{Background:} The Avogadro project has developed an advanced molecule editor and
visualizer
designed for cross-platform use in computational chemistry, molecular
modeling, bioinformatics, materials science, and related areas. It
offers flexible, high quality rendering, and a powerful plugin
architecture. Typical uses include building molecular structures,
formatting input files, and analyzing output of a wide variety of
computational chemistry packages. By using the CML file format as its native
document type, Avogadro seeks to enhance the semantic accessibility of
chemical data types.

\textbf{Results:} The work presented here details the Avogadro library, which
is a framework providing a code library and application programming
interface (API) with three-dimensional visualization
capabilities; and has direct applications to research and education in the
fields of chemistry, physics, materials science, and biology. The Avogadro
application provides a rich graphical interface using dynamically loaded
plugins through the library itself. The application and library can
each be extended by implementing a plugin module in C++
or Python to explore different visualization techniques,
build/manipulate molecular structures, and interact with other programs.
We describe some example extensions, one which uses a genetic algorithm to find
stable crystal structures, and one which interfaces with the PackMol program to
create packed, solvated structures for molecular dynamics simulations. The 1.0
release series of Avogadro is the main focus of the results discussed here.

\textbf{Conclusions:} Avogadro offers a semantic chemical builder and
platform for visualization and analysis. For users, it offers an
easy-to-use builder, integrated support for downloading from common
databases such as PubChem and the Protein Data Bank, extracting
chemical data from a wide variety of formats, including computational
chemistry output, and native, semantic support for the CML file
format. For developers, it can be easily extended via a powerful
plugin mechanism to support new features in organic chemistry,
inorganic complexes, drug design, materials, biomolecules, and simulations.
Avogadro is freely available under an open-source license from
http://avogadro.openmolecules.net.

\end{abstract}

%\ifthenelse{\boolean{publ}}{\begin{multicols}{2}}{}

\section*{Introduction}

Many fields such as chemistry, materials science, physics, and biology, need
efficient computer programs to both build and visualize molecular structures.
The field of molecular graphics is dominated by viewers with little or no
editing capabilities, such as RasMol~\cite{RasMol}, Jmol~\cite{JMol},
PyMOL~\cite{PyMOL}, VMD~\cite{VMD}, QuteMol~\cite{QuteMol},
BALLView~\cite{BALLView}, VESTA~\cite{vesta3}, and
XCrySDen~\cite{xcrysden1}\cite{xcrysden2}, among many others. The aforementioned
viewers are all freely available, and most of them are available under
open-source licenses and work on the most common operating systems (Linux, Apple
Mac OS X, Microsoft Windows, and BSD).

The choice of software capable of building chemical structures in
three dimensions is far smaller.
There are existing commercial packages, such as
CAChe/Scigress~\cite{CAChe}, ChemBio3D~\cite{Chem3D},
GaussView~\cite{GaussView}, HyperChem~\cite{HyperChem},
CrystalMaker~\cite{CrystalMaker},
Materials Studio~\cite{Accelrys}, and Spartan~\cite{Spartan},
which are polished and capable of
constructing many different types of molecular structures. They are,
however, not available for all operating systems (most of them only
run on Microsoft Windows), and are not easily extensible, customized,
or integrated into automated workflows. Licensing costs can be
prohibitive. If the company were to change its direction or focus, this
can lead to a loss of a significant research investment in a commercial
product. Furthermore, in most cases, these programs use custom,
proprietary file formats, and semantic and chemical data can be lost in
conversion to other data formats.

The selection of free, open-source, cross-platform, three-dimensional,
molecular builders was quite limited when the Avogadro project was founded in late 2006.
Ghemical~\cite{Ghemical} was one of the only projects satisfying these needs at
the time. Two of the authors (Hutchison and Curtis) contributed to Ghemical
previously, but had found that it was not easily extensible. This led them to
found a new project to address the issues they had observed in Ghemical
and other packages. The Molden~\cite{Molden} application was also available,
able to build up small molecules and analyze output from several quantum codes.
However, it suffers from a restrictive license and it uses an antiquated graphical
toolkit, which is not native on most modern operating systems.

Broad goals for the design of a molecular editor were identified following a
case study of the available applications. One of the main issues with both
commercial and open-source applications is a lack of extensibility; many of the
applications also only work on one or two operating systems. The
creation of an open and extensible framework that implements many of the necessary foundations
for a molecular builder and visualizer would facilitate more effective research
in this area. Further, the open, standardized Chemical Markup Language
(CML) file format~\cite{CML2011a, CML2011b} would be used, to secure
semantic and chemical data and allow easy interoperability with other
chemistry software.

At the time of writing, it is apparent that other researchers have perceived
similar needs. Several new applications are available today that focus on both
building and visualizing molecular structure. These include
CCP1GUI~\cite{CCP1GUI}, Gabedit~\cite{Gabedit} and some highly specific
editors such as MacMolPlt~\cite{MacMolPlt} which focus on particular
computational packages (i.e., GAMESS-US for MacMolPlt). Whilst offering many
interesting and
useful features, these projects suffer from the same issues centering
around effective reuse of existing code, well commented and documented code, and
easy extension to add new features and adapt for specialized areas.

The Avogadro project was started in earnest in 2007, and over the
first 5 years of development has been downloaded over 270,000
times~\cite{Downloads}, been translated into over 20
languages~\cite{Translations}, and has over 20
contributors~\cite{OhlohContributors}. So far, it has been cited over
100 times~\cite{ScholarCitations}, including applications in spectroscopy,
catalysis, materials chemistry, theoretical chemistry, biochemistry, and
molecular dynamics, among many
others~\cite{MeraAdasme:2011hj,Closser:2010kc,Ide:2011cj,Menegazzo:2012by,
Patel:2011fe,Popov:2011gv,Hu:2011ce,Bingol:2012kx,Yao:2010id,Fleisher:2011vz,
Mayorkas:2011eu,Tian:2011ej,Kapla:2012ho,Mandal:2012ff,Bernstein:2010kc,
Hlawacek:2011gh,Forster:2012ka,Burkhardt:2010kc,Burkhardt:2011hn,Madison:2011kw}
.

From the beginning, the project
has strived to make a robust, flexible framework for
both building and visualizing molecular structures. Much of the initial focus
has been placed on preparing input and analyzing output from quantum
calculations. Other applications such as preparing input for MD simulations and
visualizing periodic structures will also be presented, demonstrating the
flexibility of the Avogadro platform. The development team has also been
members of the Blue Obelisk movement, following the three pillars outlined by
the group: Open Data, Open Standards, and Open Source~\cite{BlueObelisk2006,
BlueObelisk2011}.

\section*{The Graphical User Interface}

The first thing most people will see is the main Avogadro application
window, as shown in Figure~1. Binary installers are provided
for Apple Mac OS X and Microsoft Windows, along with packages for all of the
major Linux distributions. This means that Avogadro can be installed quite
easily on most operating systems. Easy to follow instructions on how to compile
the latest source code are also provided on the main Avogadro web
site\cite{CompileWindows,CompileLinux} for the more adventurous, or those using
an operating system that is not yet supported.

The Qt toolkit gives Avogadro a native look and feel on the three major
supported operating systems---Linux, Apple Mac OS X, and Microsoft Windows. The
basic
functionality expected in a molecular builder and viewer has been implemented,
along with several less common features. It is very easy for new users to
install Avogadro and build their
first molecules within minutes. Thanks to the Open Babel library~\cite{OpenBabel}, Avogadro
supports a large portion of the chemical file formats that are in
common use.

The vast majority of this functionality has been written using the
interface made available to plugin writers, and is loaded at
runtime. We will discuss these plugin interfaces and descriptions of
the plugin types later.

\subsection*{Semantic Chemistry}

Avogadro has used CML~\cite{CML2011a, CML2011b} as its default file format from
a very early stage; this was chosen over other file formats because of the
extensible, semantic structure provided by CML, and the support available in
Open Babel~\cite{OpenBabel}. The CML format offers a number of advantages over
others in common use, including the ability to extend the format. This allows
Avogadro and other programs to be future-proof, adding new information and
features necessary for an advanced semantically-aware editor at a later time,
while still remaining readable in older versions of Avogadro.

Through the use of Open Babel~\cite{OpenBabel}, a large array of file formats
can be interpreted. When extending Avogadro to read in larger amounts of the
output from quantum codes, it was necessary to devote significant development
resources to understanding and adding semantic meaning to the quantum code
output. This work was developed in a plugin, which was later split out into a
small independent library called OpenQube~\cite{OpenQube,OpenQubeSource}. More
recently a large amount of work has been done by the Quixote
project~\cite{Quixote}, JUMBO-Converters, and the Semantic Physical Science
workshop to augment quantum codes to output more of this data directly from the
code. Since CML can be extended, it is possible to reuse existing conventions
for molecular structure data, and add new conventions for the additional quantum
data.

\subsection*{Building a Molecule: Atom by Atom}

After opening Avogadro a window such as that shown in Figure~1
is presented. By default, the draw tool is selected. Simply left-clicking on the
black part of the display
allows the user to draw a carbon atom. If the user pushes the left
mouse button down and drags, a bonded carbon atom is drawn
between the start point and the final position where the mouse is released.

A large amount of effort has been expended to create an intuitive tool for
drawing small molecules. Common chemical elements can be selected from a drop
down list,
or a periodic table can be displayed to select less common elements. Clicking on
an existing atom changes it to the currently selected element,
dragging changes the atom back to its previous element and draws
a new atom bonded to the original. If the bonds are left-clicked then the bond
order cycles between single, double, and triple. Shortcut keys are
also available, e.g., typing the
atomic symbol (e.g., ``C-o'' for cobalt) changes the selected element,
or typing the numbers ``1,'' ``2,'' and ``3'' changes the bond order.

Right clicking on atoms or bonds deletes them. If the ``Adjust Hydrogens'' box
is checked, the number of hydrogens bonded to each atom is automatically
adjusted to satisfy valency. Alternatively, this can also be done at the end of
an editing session by using the ``Add hydrogens'' extension in the build menu.

In addition to the draw tool, there are two tools for adjusting the position of
atoms in existing molecules. The ``atom centric manipulate'' tool can be used to
move an atom or a group of selected atoms. The ``bond centric manipulate'' tool
can
be used to select a bond, and then adjust all atoms positions relative to the
selected bond in various ways (e.g., altering the bond length, bond
angles, or dihedral angles). These three tools allow for a great deal of
flexibility in building small molecules interactively on screen.

Once the molecular structure is complete, the force field extension can be used to
perform a geometry optimization. By clicking on ``Extensions'' and ``Optimize
Geometry'' a fast geometry optimization is performed on the molecule. The
force field and calculation parameters can be adjusted, but the defaults are
adequate for most molecules. This workflow is typical when building up a small
molecular structures for use as input to quantum calculations, or publication
quality figures.

An alternative is to combine the ``Auto Optimization'' tool with the drawing
tool. This presents a unique way of sculpting the molecule while the geometry is
constantly minimized in the background. The geometry optimization is animated,
and the effect of changing bond orders, adding new groups, or removing groups can
be observed interactively.

Several dialogs are implemented to provide information on molecule properties
and to precisely change parameters, such as the cartesian coordinates of the atoms
in the molecule.

\subsection*{Building a Molecule: From Fragments}

In addition to building molecules atom-by-atom, users can insert
pre-built fragments of common molecules, ligands, or amino-acid
sequences, as shown in Figure~2.
In all cases, after inserting the fragment, the atom-centered manipulate tool
is selected, allowing the fragment to be moved or rotated into
position easily.

Users can also insert a SMILES~\cite{smiles,opensmiles} string for a molecule. In
this case, a rough 3D geometry is generated using Open Babel and a
quick force field optimization.

\subsection*{Preparing Input for Quantum Codes}

Several extensions were developed for Avogadro that assist the user in preparing
input files for popular quantum codes such as
GAMESS-US,~\cite{GAMESS-US} NWChem,~\cite{NWChem}
Gaussian,~\cite{g09} Q-Chem,~\cite{Q-Chem3} Molpro,~\cite{MOLPRO}
and MOPAC200x~\cite{MOPAC}. The graphical dialogs present the features
required to run basic quantum calculations; some examples are shown in
Figure~3.

The preview of the input file at the bottom of each dialog is updated as options
are changed. This approach helps new users of quantum codes to learn the syntax
of input files for different codes, and to quickly generate useful input files
as they learn. The input can also be edited by hand in the dialog before the file is saved
and submitted to the quantum code. The MOPAC extension can also run the
MOPAC200x
program directly if it is available on the user's computer, and then reload the
output file into Avogadro once the calculation is complete. This
feature will be extended to other quantum codes in future versions of Avogadro.

The GAMESS-US plugin is one of the most highly developed, featuring a basic dialog
present in most of the other input deck generators, as well as an advanced dialog
exposing many of the more unusual and complex calculation types. In addition to
the advanced dialog, the input deck can be edited inline and features syntax
highlighting (Figure~4) as used in many popular editors aimed at software developers. This
can indicate simple typing errors in keywords, as well as harder to spot whitespace
errors that would otherwise cause the hand-edited input deck to fail when being
read by GAMESS-US.

\subsection*{Alignment and Measurements}

One of the specialized tools included in the standard Avogadro distribution is
the alignment tool. This mouse tool facilitates the alignment of a molecular
structure with the coordinate origin if one atom is selected, and along the
specified axis if two atoms are selected. The alignment tool can be combined
with the measure, select, and manipulate tools to create inputs for
quantum codes where the position and orientation of the molecule is important.
One example of this is calculations where an external electric field is applied
to the molecule. In these types of calculations, the alignment of the molecule
can have a large effect. Figure~5 shows the measurement tool
in action with the alignment tool configuration dialog visible in the lower-left
corner.

More complex alignment tools for specific tasks could be created. The alignment
tool was created in just a few hours for a specific research project. This is a prime
example where extensibility was very important for performing
research using a graphical computational chemistry tool. It would not be
worth the investment to create a new application just to align molecular
structures to an axis, but creating a plugin for an extensible project is not
unreasonable.


\section*{Visualization}

The Avogadro application uses OpenGL to render molecular representations to the
screen interactively. OpenGL offers a high-level, cross-platform API for rendering
three-dimensional images using hardware accelerated graphics. OpenGL
1.1 and below is used in most of the rendering code, and so Avogadro can be used
even on older computer systems, or those without more modern
accelerated graphics. It is capable of taking advantage of some of the
newer features available in OpenGL 2.0 as described below, such as but
this has been kept as an optional extra feature when working on novel
visualizations of molecular structure.

\subsection*{Standard Representations}

In chemistry, there are several standard representations of molecular structure,
originally based upon those possible with physical models. The Avogadro
application implements each of these representations shown in
Figure~6 as a plugin. These range from the simple wireframe
representation, stick/licorice, ball and stick, and Van der Waals spheres.

It is also possible to combine several representations, such as ball and stick
with ring rendering (Figure~6 (d)), and a semi-transparent Van
der Waals space-filling representation with a stick representation to elucidate
molecular backbone (Figure~6 (f)).

\subsection*{Electronic Structure}

Quantum codes were originally developed for line printers, and unfortunately little has
changed since then in the standard log files. There are several formats developed for
use in other codes and specifically for visualization and analysis, but there is little
agreement on any standard file format in the computational quantum chemistry
community. A plugin was developed in Avogadro to visualize the output of various
quantum codes, and get the data into the right format for further visualization and
analysis.

Initially support was added and extended in Open Babel for Gaussian cube files.
This format provides atomic coordinates and one or more regularly spaced grids
of scalar values. This can be read in, and techniques such as the marching cubes
algorithm can be used to compute triangular meshes of isosurfaces at values of
electron density for example. Once the code has been developed to visualize these
isosurfaces, it became clear that it would be useful to be able to calculate these cubes
on the fly, and at different levels of detail depending upon the intended use.

The first format, which was somewhat documented at the time it was developed, is the
Gaussian formatted checkpoint format. This format is much easier to parse than the
log files generated as the program runs, and provides all of the detail needed to
calculate scalar values of the molecular orbital or electron density at any point in space.
Once a class structure had been developed for Gaussian type orbitals, the approach
was extended to read in several other popular output file formats including Q-Chem,
GAMESS-US, NWChem, and Molpro. MOPAC200x support was added later, along with
support for the AUX format and Slater type orbitals used in that code. All of these
codes output their final configurations using the standard linear combination of
atomic orbitals, meaning that parallelization is extremely simple.

The plugin was developed to take advantage of the map-reduce approach offered
by QtConcurrent in order to use all available processor cores. This offers almost
linear scaling as each point in the grid can be calculated independently of all other
points, the results of which can be seen in Figure~7.
An alternate approach to calculating the molecular orbitals was developed
in a second plugin that has since been split off into a separate
project named ``OpenQube''. The ``OpenQube'' library has also been added as an
optional backend in VTK during the 2011 Google Summer of Code, bringing support
for several output file formats and calculation of cube files that can later be
fed into more advanced data pipelines.

There are several related projects for adding semantic meaning to this type of output,
including the JUMBO-Converters project and Quixote. It is hoped that more codes
will adopt semantic output in the future, using a common format so that data exchange,
validation, and analysis become easier across several codes. This was the subject of
a recent meeting with several computational chemistry codes beginning to use FoX
in order to output CML. Development has begun on code to read in CML output,
either directly from the codes or from conversion of other formats using Open Babel
or the JUMBO-Converters. If enough semantic structure can be added to CML, and the
converters support a large enough range of the output, this could replace most of
the parsing code present in OpenQube. Semantic meaning is one of the most
difficult to extract from log files, and coming together as a community will help
projects like Avogadro to derive more meaning from the outputs of these codes.

\subsection*{Secondary Biological Structure}

Avogadro uses the PDB reader from Open Babel to read in the secondary
biological structure. Two plugins exist to process and render this information.
The first is plugin which renders a simple tube between the biomolecule
backbone atoms. A second more advanced plugin calculate meshes for
the alpha helices and beta sheets. While the first plugin is much
faster, the advanced plugin more accurately produces output expected
in the field.  This allows users flexibility for rendering
secondary biological structures.

\subsection*{GLSL, Novel Visualization}

GLSL, or OpenGL Shader Language, is a C-like syntax that can be used to develop
code that will run on graphics cards and included in the OpenGL 2.0
specification. It has been used to great effect by the games industry,
as well as in many areas of data visualization. Several recent papers
highlight the potential in chemistry, such as QuteMol~\cite{QuteMol}
in adding support for features such as ambient occlusion to add depth to images.

Avogadro has support for vertex and fragment shader programs, and several
examples are bundled with the package. If the user's graphics card is capable,
these programs can be loaded at runtime and used to great effect to visualize
structure. Some of these include summarization techniques such as isosurface
rendering where only the edges orthogonal to the view plane are visible, giving
a much better rendering of both the molecular and electronic structure (Figure~8).

\subsection*{Ray Tracing}

Avogadro uses a painter abstraction that makes it much easier for developers to
add new display types. It also abstracts away the renderer, making it possible
to add support for alternative backends. Currently only OpenGL and POV-Ray are
supported. Due to the abstraction, we are able to use the implicit surfaces
available in ray tracers to render molecular structure at very high levels of
clarity and with none of the triangle artifacts present in standard OpenGL rendered
images. Much higher quality transparency and reflection also allow for the
images to be used in poster and oral presentations as well as research
articles (Figure~9).

This feature is implemented in an extension, with an additional painter class
deriving from the base class and a dialog allowing the user to edit the basic
rendering controls. The POV-Ray input file can also be retained and edited to
produce more complex images, or to allow for much finer control of the
rendering process if desired.

\section*{Software Architecture}

One area that seems to suffer in many code bases in chemistry is software
architecture. This can lead to less maintainable code, poor code reuse, and a
much higher barrier to entry. Problems were identified in other projects with a
view to minimize their impact when developing Avogadro. Modern software design
processes were used in the initial planning stages of Avogadro, along with the
choice of modern programming languages and libraries.

Avogadro has close ties to several other free, cross-platform, open-source
projects to reuse as much code as is practical. These projects include
Qt~\cite{Qt} to provide a free, cross-platform graphical toolkit; Open
Babel~\cite{OpenBabel} for chemical file input/output, geometry optimization, and
other chemical perception; Eigen~\cite{Eigen} for matrix and vector mathematics;
OpenGL/GLSL for real-time, three-dimensional rendering; and POV-Ray for ray-traced
rendering.

Based on the previous experience of the authors and a review of
available programs at the time, several fundamental choices were made.
The C++ programming language; the Qt graphical toolkit;
OpenGL for 3D visualization; CMake as the build system; and Open Babel as
the chemical library. Using this combination of languages and
libraries requires the project to be licensed under the GNU GPLv2~\cite{GPLv2}
license and made openly available to all.

The core of Avogadro is written in portable C++ code with platform-specific
differences abstracted away by Qt, OpenGL, and Open Babel. The CMake build system
makes the build process relatively simple on all supported platforms. Avogadro
has been successfully built and tested on Linux, Apple Mac OS X, and Microsoft
Windows in common 32 and 64 bit hardware architectures.

The Avogadro framework uses the model, view, controller paradigm. The model
is comprised of the core data classes such as {\tt Molecule}, {\tt Atom}, and
{\tt Bond}, views are made up of the engine/display plugins, and
controllers are the tools (interactive mouse) and extensions
(non-interactive, form based/menu based). Every plugin has full access
to the core data model, but view and controller plugins are
conceptually different; views are responsible for displaying
data and controllers are responsible for modifying/changing data.

Plugins rely on Avogadro's set of programming interfaces and almost
all functionality is implemented in self contained plugins that are
loaded at runtime. The majority of plugins distributed with Avogadro
are written in C++, but the API is also available in the Python
scripting language. This allows for a great deal of choice in how
plugins are implemented. Each plugin is a singleton class that
implements a particular set of functions--depending on the type of
plugin--which allows for features to be implemented in a very modular
way.

Over the last few years Avogadro development has started to use nightly builds
of the latest version of the code in order to automatically flag issues
introduced in new commits. Code review was also introduced in order to add a
review step before new code is merged, along with softening the line between
someone with commit rights and someone without (anyone can propose and upload
a patch, but a small group can choose if/when the patch will be merged). Some
automated testing has been added, but coverage at this point remains relatively
low. API documentation is automatically generated from comments in the code
using Doxygen.

\subsection*{Plugin Interface}

Avogadro plugins are divided into four different types corresponding
to four main classes
that derive from this common base class, specializing
their interface for specific activities
(Figure~10). The {\tt Avogadro::Color} base class
defines the virtual interface for applying colors to atoms, bonds, and other properties.
 {\tt Avogadro::Engine} defines the common interface for all display types in Avogadro:
simple ball and stick, Van der Waals visualizations,
surfaces, and force visualizations. The {\tt Avogadro::Tool} base class provides
the interface for all interactive tools, focusing principally on mouse and
keyboard interaction with Avogadro. Examples of tool plugins include the draw
tool used to draw molecules atom by atom, and the navigation tool used to pan,
rotate, and scale the view of the molecule. There are also several specialized
tools such as the alignment tool.

Finally there is the {\tt Avogadro::Extension} class, which defines the interface for
dialog based plugins. These extensions can interact with the molecule, and are
used for a variety of purposes from molecule properties dialogs to input file
generation dialogs for many quantum codes including NWChem, Gaussian, GAMESS, and
others. This class of plugin is also applied to file import, and network aware
extensions querying web databases for structures given their common name for
example.

At start up, several standard directories, which may be customized, are searched
for plugins. The Qt plugin framework is used to check that the plugins have a
recent enough version to be loaded, and the plugin type can be deduced once
loaded. The user interface is then populated with appropriate entries; tools are
added to the main toolbar using their embedded icons, display types are added to
the display type list, and menu entries are added for all loaded extensions.

The tool and display type plugins can both (optionally) provide a
dialog for configuring the plugin. Dialogs are specific to each plugin
and integrated into the user interface.

\subsection*{Display Types}

Display plugins are referred to as ``engines'' internally. Their
primary focus is rendering graphics to the screen. As is the case with most
molecular graphics, a large portion of the geometric primitives are spheres and
cylinders, typically used to represent atoms and bonds. There are many other
properties that can be rendered using the display type plugins, for example,
some of the engines also convey information about the underlying data the
geometric primitives represent to allow for the molecule to be edited. Table~1
shows a summary of the display plugins distributed with Avogadro.

Engines are performance critical as the render functions are called each time a
frame is requested for display.  Efficient rendering is also critical
since multiple display types can be combined to form a composite display. For
example, ball and stick display overlaid with a transparent Van der Waals
space-filling display and ring rendering to highlight all rings in the
structure. Figure~6 (d) and (f) show two such combinations of
multiple display types.

\subsection*{Tools}

The tools are responsible for virtually all mouse and keyboard interaction with
the molecule. A list of all tools is given in Table~2.

The navigation tool provides basic scene navigation, implementing rotation,
panning, tilting, and zooming support. The initial point of interaction (where
the click occurs) changes the anchor point for navigation; navigation takes
place about the center of molecule when clicking in empty space or about the
center of any clicked atom. During interaction, the navigation tool provides
visual cues to show what type of navigation is taking place. The navigation
tool is also used as the default tool if the currently active tool does not
handle the mouse event passed to it.

One of the other central tools is the draw tool, which implements a free-hand
molecule drawing input method supporting keyboard shortcuts, combo boxes, and a
periodic table view to select elements. The user can use the left mouse button
to add new atoms or bonds, or click on the bonds to change their order. The
right mouse button can be used to delete atoms or bonds, and the directional keys
can be used in combination with the mouse to quickly rotate/pan the molecule.

There are also two tools for adjustment of structures (atom or bond centric), a
selection tool supporting standard selection interactions, and an auto-rotate
tool that allows users to set the speed and angles about which to rotate the
molecule. The interactive auto-optimization tool provides a sculpting
interaction, where the user can begin a continuous geometry optimization and
switch back to the draw or adjustment tools and change the shape and structure
of the molecule while observing the new structure being optimized.

This can also be combined with the measurement tool to interactively observe
bond lengths and angles evolve as the structure is updated and the geometry
minimized. If the optimization tool is turned off, the measurement tool also
allows the user to precisely adjust bond lengths and/or angles using the
adjustment tools.

\subsection*{Extensions}

Extensions represent quite a diverse range of plugins including
input generation dialogs for various quantum chemistry codes such as GAMESS,
Molpro, NWChem, etc., animation of the molecule, and visualization of
molecular orbitals and electron density. Network aware extensions allow the user
to click on a menu item to fetch by chemical name and search for ``tnt'' or
``propanol'' and have structures returned by the NIH CACTUS Chemical Structure
Resolver service~\cite{StructureResolver}. A summary of the extensions
distributed with Avogadro is shown in Table 3.

Other extensions translate the entire scene to POV-Ray input, and call POV-Ray
to render the molecule using ray tracing techniques to provide higher quality
renderings for publication. Various molecular property dialogs are also
implemented as plugins, drawing largely on Open Babel functionality to provide
an overview of the molecule. Cartesian editors, addition and removal of
hydrogens, fragment, SMILES, and peptide insertion are all implemented as
extensions showing up in Avogadro menus. More recently a crystallography
extension was added, giving access to a much wider range to functionality
useful to practitioners in that area, including Miller Plane
visualization, slab and surface generation. New builders for
nanotubes, nanoparticles, and DNA are also planned for upcoming releases.

\subsection*{Colors}

The color plugins primarily take either double precision numbers or integer
values and return an RGB value. The plugins range from the standard color
plugin that takes atomic number and returns the standard RGB value for that
element through to mapping things like partial change and index to more easily
view various aspects of the molecule's structure.

By defining a plugin interface for coloring atoms, bonds, or residues,
developers can easily offer flexible rendering options to highlight
important information without requiring a user to tediously set colors
on specific atoms or functional groups. Default color plugins are
listed in Table~4, illustrating the variety of
options. Each plugin is usually only 40-50 lines of C++ code.

\section*{Python Interface}

Python bindings are provided for all of the core API. Python code can be used
in two ways: the first is the interactive Python terminal, and the second is to
write Python plugins; extensions, tools, or display types.
Writing a Python plugin requires the same functionality to be implemented as a
native C++ plugin~\cite{PythonExtensions}. The advantage of Python
plugins is that it's easier to make prototypes since no compilation is
required. Python plugins can also easily be shared with other users.

The Python bindings interface with the PyQt python bindings for the Qt
toolkit, which enables Python code to use all of Qt's features when writing a
plugin. For example, a short Python script can present a window using Qt and
render molecules using Avogadro~\cite{PythonWindowExample,PythonScripting,PyQtGist}.

Avogadro also includes an interactive Python console
(Figure~11, which allows users to directly script
and manipulate the Avogadro environment~\cite{PythonTerminalTutorial}.


\section*{Quantum Calculations} % Marcus
% Is this a duplicate of the "Electronic Structure" section above?

Once quantum calculations have been performed, it is desirable
to visualize various properties including the molecular orbitals and electron
density of the final state of the system. The OpenQube library houses most of
the code that reads in the relevant matrices and factors and takes
care of translating output into a
standard format as expected by the calculation code.

Once the data has been parsed and normalized, initialization of data
structures is required and these structures are treated as read-only. This
allows the calculation to be multithreaded and share one copy of the input data.
The scalar value of the molecular orbital or electron density is calculated as a
linear combination or atomic orbitals. The parallelization takes place across a
regularly spaced cube with each point being calculated in an independent thread
sharing the input arrays and matrices. The QtConcurrent framework abstracts away
thread pool management, using as many cores as available on the host system.

Once a cube has been calculated, it is passed to the marching cubes
algorithm which calculates an isosurface. Each isosurface is
calculated in a separate thread; at this stage the cube is read only,
and so both the positive and negative isosurface of a molecular
orbital can be calculated in separate threads. Further parallelization
of this stage is possible, but typical isosurfaces do not take long to
calculate. This could be moved to the GPU as it is likely to be more useful
because the result would already be in GPU memory for rendering.

A class hierarchy with a standard API is provided for quantum output. Adding
support for new codes involved developing a new parser and ensuring the
Gaussian or Slater set is populated with the correct ordering and the expected
normalization scheme. The s, p, and d-type Gaussian orbitals are supported,
with f and g support planned in order to support the increasing number of
calculations using these higher-order orbitals. The Basis Set Exchange hosted
by EMSL provides access to the basis sets in common use, although at present
these basis sets are normally read in directly from the output files.

\section*{Avogadro Library in Use}

The Avogadro library's first use was the Avogadro application, closely
followed by the Kalzium periodic table program that is part of the KDE software
collection. This initial work was funded in part by the Google Summer of Code
program in 2007, and also resulted in the addition of several other features in
the Avogadro library to support Kalzium and general visualization and editing
of molecular structure (Figure~12).

The Q-Chem package~\cite{Q-Chem3} has developed ``QUI - The Q-Chem User
Interface''~\cite{QUI} around Avogadro, originally as an Avogadro extension.
This is a more advanced version of the input generator developed in Avogadro,
with much tighter integration. Molpro~\cite{MOLPRO} has also published some
results from their development of a Molpro interface using the Avogadro
library~\cite{MOLPROGUI}.

\subsection*{Packmol}

Packmol is a third-party package designed to create initial
``packed'' configurations of molecules for molecular dynamics or other
simulations~\cite{packmol,packmol-packing}. Examples include
surrounding a protein with solvent, solvent mixtures, lipid bilayers,
spherical micelles, placing counterions, adding ligands to
nanoparticles, etc.

Typically, users may have equilibrated ``solvent boxes'' which have
been run for long simulations to ensure proper density, and both short
and long-range interactions between solvent molecules. Using such
solvent boxes allows placing solute molecules, such as proteins, in an
approximately correct initial structure, such as that shown in Figure~13.
The solute is added into the
box, and solvent molecules with overlapping atoms are removed. While
these utilities are often enough, creating complex input files is not
always easy.

For more complicated systems, Packmol can create an initial
configuration based on defined densities, geometries (e.g., sphere,
box, etc.), and the molecules to be placed. An Avogadro developer wrote
an external plugin to facilitate use of Packmol, including estimating
the number of molecules in a given volume.

The plugin is not currently distributed with Avogadro as a standard
feature, although it is planned for some future version. It serves as
an example of how Avogadro can facilitate a workflow with a
text-oriented package (Packmol), including saving files in the PDB
format required by Packmol, generating an input file, and reading the
output for visualization, analysis, and further simulations.

\subsection*{XtalOpt}

The XtalOpt~\cite{xo1, xo2} software package is implemented as a third-party C++
extension to Avogadro and makes heavy use of the libavogadro API. The extension
implements an evolutionary algorithm tailored for crystal structure prediction.
The XtalOpt development team chose Avogadro as a platform because of its
open-source license, well-designed API, powerful visualization tools, and
intuitive user-interface. XtalOpt exists as a dialog window
(Figure~14) and uses the main Avogadro window for visualizing
candidate structures as they evolve. The API is well suited for XtalOpt’s needs,
providing a simple mechanism to allow the user to view, edit, and export the
structures generated during the search. Taking advantage of the cross-platform
capabilities of Avogadro and its dependencies, XtalOpt is available for Linux,
Windows, and Mac.

\section*{Conclusions and Future Directions}

Avogadro has grown over its first six years to become an important
tool for building, editing, visualizing, and analyzing chemical and
molecular data. With over 270,000 downloads,
language translations and localizations, and over 100 citations, it
has become an integral part of the chemical software toolbox. Through
use of the native CML file format and a wide variety of chemical data
import, Avogadro can provide semantic chemical data editing and conversion.
We seek to provide an integrated environment in the simulation and
cheminformatics workflow. While more must be done, particularly in
regards to documentation, tutorials, ease-of-use, and automation, we
aim to improve the quality and feature set with each new release.

Currently, two upcoming versions of Avogadro are under development. The
first is Avogadro version 1.1, which adds additional features and
refinement, particularly including crystallography support developed
through the XtalOpt project. The second is a more substantial
development for Avogadro version 2.0, where many of the core data
structures are being rewritten in order to offer greater flexibility and
scalability. Our goal is to support an increasing scope of chemical
systems, including biomolecules (DNA, RNA, saccarides, etc.),
materials (crystallography, polymers, surfaces), nanoscience
(nanoparticles, nanotubes, graphene, etc.) with improved speed, intuitive
ease-of-use and simpler non-reciprocal licensing terms.

Avogadro is freely available from http://avogadro.openmolecules.net/,
and new contributors are welcome in all areas (users, developers,
testers, translators, educators, students, researchers, dreamers).

\section*{Availability and Requirements}

\textbf{Project Name:} Avogadro \\
\textbf{Project home page:} http://avogadro.openmolecules.net/ \\
\textbf{Operating system(s):} Cross-platform \\
\textbf{Programming language:} C++, bindings to Python \\
\textbf{Other requirements (if compiling):} CMake 2.6+, Open Babel, Qt 4.6+,
Eigen 2 \\
\textbf{License:} GNU GPL v2 \\
\textbf{Any restrictions to use by non-academics:} None additional

\section*{Acknowledgements and Funding}

We wish to thank the many contributors to the Avogadro project,
including developers, testers, translators, and users. We thank
SourceForge for providing resources for issue tracking and managing
releases, Launchpad for hosting language translations, and Kitware for
additional dashboard resources. MDH and GRH thank the University of
Pittsburgh for support. DEC would like to thank Jan Halborg Jensen for
designing the GAMESS-US interface and supporting Avogadro in its
infancy; believing Avogadro could be better than what was available.
MDH acknowledges the Engineering Research Development Center (W912HZ-11-P-0019)
for financial support. EZ and DL acknowledge the NSF (DMR-1005413) for financial
support.

\section*{Authors' contributions}
GRH and DEC are the founders of the Avogadro project. MDH is the
current lead developer and maintainer of Avogadro. GRH, DL and TV are
active developers. DL and EZ are founders of the XtalOpt project which
is discussed in this work. TV developed the PackMol plugin. All
authors read and approved the final manuscript.

\section*{Competing interests}

The authors declare that they have no competing interests.

%%%%%%%%%%%%%%%%%%%%%%%%%%%%%%%%%%%%%%%%%%%%%%%%%%%%%%%%%%%%%
%%                  The Bibliography                       %%
%%                                                         %%
%%  Bmc_article.bst  will be used to                       %%
%%  create a .BBL file for submission, which includes      %%
%%  XML structured for BMC.                                %%
%%  After submission of the .TEX file,                     %%
%%  you will be prompted to submit your .BBL file.         %%
%%                                                         %%
%%                                                         %%
%%  Note that the displayed Bibliography will not          %%
%%  necessarily be rendered by Latex exactly as specified  %%
%%  in the online Instructions for Authors.                %%
%%                                                         %%
%%%%%%%%%%%%%%%%%%%%%%%%%%%%%%%%%%%%%%%%%%%%%%%%%%%%%%%%%%%%%

\newpage
{\ifthenelse{\boolean{publ}}{\footnotesize}{\small}
 \bibliographystyle{bmc_article}  % Style BST file
  \bibliography{AvogadroPaper} }     % Bibliography file (usually '*.bib' )

%%%%%%%%%%%

\ifthenelse{\boolean{publ}}{\end{multicols}}{}

%%%%%%%%%%%%%%%%%%%%%%%%%%%%%%%%%%%
%%                               %%
%% Figures                       %%
%%                               %%
%% NB: this is for captions and  %%
%% Titles. All graphics must be  %%
%% submitted separately and NOT  %%
%% included in the Tex document  %%
%%                               %%
%%%%%%%%%%%%%%%%%%%%%%%%%%%%%%%%%%%

%%
%% Do not use \listoffigures as most will included as separate files

\section*{Figures}
  \subsection*{Figure 1 - The Avogadro graphical user interface}
    Taken on Mac OS X, showing the editing interface for a molecule.

  \subsection*{Figure 2 - Dialogs for inserting pre-built fragments}
    The left shows molecules, and the right amino-acid sequences.

  \subsection*{Figure 3 - Dialog for generating input for quantum codes}
    Dialogs for generating input for Q-Chem, NWChem, Molpro and MOPAC200x.
    Note that the dialogs are similar in interface, allowing users to use
    multiple computational chemistry packages.

  \subsection*{Figure 4 - The GAMESS-US input deck generator}
    This input generator has an advanced panel and syntax highlighting.

  \subsection*{Figure 5 - The measurement tool}
    The measurement tool being used to measure bond angles and lengths
    (on Linux with KDE 4).

  \subsection*{Figure 6 - Standard molecular structure representations}
    Several molecular representations of thiophene, (a) wireframe,
    (b) stick/licorice, (c) ball and stick, (d) ball and stick with ring,
    (e) Van der Waals/CPK and (f) transparent Van der Waal's with stick.

  \subsection*{Figure 7 - Molecular orbitals and surfaces}
    Rendering of a molecular orbital isosurface (left) and an electrostatic
    surface potential mapped onto the electron density (right).

  \subsection*{Figure 8 - Molecular orbitals rendering using GLSL shaders}
    Rendering of a molecular orbital isosurface using two GLSL shaders to
    highlight the edges of the surfaces. The X-ray effect (left) and red and blue (right)
    showing the positive and negative molecular orbital shapes.

  \subsection*{Figure 9 - Ray-traced HOMO isosurfaces of varying cube density}
    Rendering of a molecular orbital isosurface using POV-Ray with cubes
    of low (left) and high (right) density.

  \subsection*{Figure 10 - General code architecture of Avogadro}
    General code architecture of Avogadro, indicating major
    plugin interfaces for colors, display engines, tools, and
    extensions. Red boxes indicate code dependencies of Avogadro, blue
    boxes indicate plugin API classes, and green boxes inidicate examples of
    each plugin type.

  \subsection*{Figure 11 - Python scripting terminal, printing atomic numbers}

  \subsection*{Figure 12 - The Kalzium application in KDE using Avogadro to render molecular structures}

  \subsection*{Figure 13 - The PackMol lipid layer as produced by the PackMol extension}

  \subsection*{Figure 14 - The XtalOpt extension}
    XtalOpt extension showing a plot of stability vs. search progress
    for a $\mathrm{TiO_2}$ supercell.


%%%%%%%%%%%%%%%%%%%%%%%%%%%%%%%%%%%
%%                               %%
%% Tables                        %%
%%                               %%
%%%%%%%%%%%%%%%%%%%%%%%%%%%%%%%%%%%

%% Use of \listoftables is discouraged.
%%
\section*{Tables}
  \subsection*{Table 1 - List of default display type (engine) plugins}
    \par
    \mbox{
\begin{tabular}{l | l}
\hline
Name & Description \\
\hline
Axes & Renders x, y, z Cartesian axes from the origin \\
Ball and Stick & Standard ball and stick representation \\
Cartoon & Secondary biological structure ($\alpha$ helix and $\beta$ sheet) \\
Dipole & Render direction/magnitude of dipole moment if present \\
Force & Renders arrows showing forces on atoms from force field \\
Hydrogen Bond & Renders hydrogen bonds as dotted lines \\
Label & Shows labels on atoms and bonds, configurable \\
Overlay & Overlay of color gradient used for electrostatic properties \\
Polygon & Renders closed polygons of metallic centers \\
Ribbon & Basic secondary structure ribbon rendering \\
Ring & Renders rings in structure, different colors depending on ring size \\
Simple Wireframe & Very simple wireframe display \\
Sticks & Stick or liquorice rendering style for atoms and bonds \\
Surface &Renders triangular isosurface meshes \\
Van der Waals Spheres & Van der Waals sphere rendering (no bonds, space-filling) \\
Wireframe & Wireframe with more features such as bond order rendering \\
\hline
\end{tabular}
      }

  \subsection*{Table 2 - List of default mouse tool plugins}
    \par
    \mbox{
\begin{tabular}{l | l}
\hline
Name & Description \\
\hline
Draw Tool & Build and edit atoms\\
Navigate Tool & Move the camera, rotate, pan, and zoom \\
Bond Centric Manipulate Tool & Alter bond lengths, angles, and torsions \\
Manipulate Tool & Move atoms and selected fragments \\
Select Tool & Select individual atoms, bonds, or fragments \\
Auto Rotate Tool & Continuously rotate a molecule for presentations \\
Auto Optimize Tool & Continuously optimize molecular geometry using molecular mechanics \\
Measure Tool & Determine bond lengths, angles, and dihedrals \\
Align Tool & Rotate and translate to a specified frame of reference \\
\hline
\end{tabular}
      }

  \subsection*{Table 3 - List of default extension commands}
    \par
    \mbox{
\begin{tabular}{l | l}
\hline
Name & Description \\
\hline
Create Surfaces & Create surface meshes from molecular orbital/electron density data \\
GAMESS & Prepare input files for GAMESS-US, featuring syntax highlighting, advanced properties \\
Insert Fragment & Insert molecular fragments from a library of common fragments \\
Insert Peptide & Build up and insert peptide fragments \\
Molecular Mechanics & Use Open Babel's force fields for geometry optimization and conformer searches \\
MOPAC & Prepare input for and run MOPAC200x  \\
POV-Ray & Ray-trace the displayed structure using POV-Ray \\
\textbf{Properties} \\
Angle Properties & Table of all bond angles (editable)\\
Atom Properties & Table of all atoms with common properties \\
Bond Properties & Table of all bonds with common properties \\
Molecule Properties & Common properties of the molecule (including
molecular weight, etc.)\\
Torsion Properties & Table of all dihedral angles (editable) \\
Spectra & Visualize spectra from output files \\
Super Cell Builder & Expand atoms with space group, replicate specified repeats and perform simple bonding \\
Unit Cell & Change crystallographic unit cell display and parameters \\
Vibrations & Show and animate molecular vibrations \\
\hline
\end{tabular}
      }

  \subsection*{Table 4 - List of default color plugins}
    \par
    \mbox{
\begin{tabular}{l | l}
\hline
Name & Description \\
\hline
Atom Index Color & Color based on atom ID (from atom 1, 2, etc.) \\
Charge Color & Color based on predicted electrostatic partial charge \\
Custom Color & Color all atoms a specific, custom color \\
Distance Color & Color based on distance from one end of the
molecule \\
Element Color \textbf{(Default)} & Standard color scheme, giving each
atom a color defined by its element\\
Residue Color & Color based on amino acid or nucleic acid residue (i.e., glycine,
histidine, etc.) \\
SMARTS Color & Color atoms matching a specific SMARTS pattern with a
custom color \\
\hline
\end{tabular}
      }

\end{bmcformat}
\end{document}
